% !TEX root = Projektdokumentation.tex
\section{Anhang}
\subsection{Detaillierte Zeitplanung}
\label{app:Zeitplanung}

\tabelleAnhang{ZeitplanungKomplett}

\input{Anhang/AnhangLastenheft.tex}
\clearpage

\subsection{Use Case-Diagramm}
\label{app:UseCase}
Use Case-Diagramme und weitere \acs{UML}-Diagramme kann man auch direkt mit \LaTeX{} zeichnen, siehe \zB \url{http://metauml.sourceforge.net/old/usecase-diagram.html}.
\begin{figure}[htb]
\centering
\includegraphicsKeepAspectRatio{UseCase.pdf}{0.7}
\caption{Use Case-Diagramm}
\end{figure}

\input{Anhang/AnhangPflichtenheft.tex}

\subsection{tt\_content.php}
\label{tt_content.php}

\begin{lstlisting}[language=json,firstnumber=1]
<?php declare(strict_types=1);

defined('TYPO3_MODE') || die();

// static TypoScript
(static function () {
    \TYPO3\CMS\Core\Utility\ExtensionManagementUtility::addPlugin(
        array(
            'LLL:EXT:bal_skeleton/Resources/Private/Language/Tca.xlf:bal_column.wizard.title',
            'tx_bal_column',
            'EXT:bal_skeleton/Resources/Public/Icons/ContentElements/stage.png'
        ),
        'CType',
        'bal_skeleton'
    );
    $temporaryColumn = array(
        'tx_bal_column_color' => array (
            'exclude' => 1,
            'label' => 'LLL:EXT:bal_skeleton/Resources/Private/Language/Tca.xlf:bal_column.color.title',
            'config' => array (
                'type' => 'input',
                'renderType' => 'colorpicker',
                'size' => 10,
            )
        ),
    );

    \TYPO3\CMS\Core\Utility\ExtensionManagementUtility::addTCAcolumns(
        'tt_content',
        $temporaryColumn
    );

    $GLOBALS['TCA']['tt_content']['types']['tx_bal_column'] = array(
        'showitem' => '
            --palette--;LLL:EXT:frontend/Resources/Private/Language/locallang_ttc.xml:palette.general;general,
            tx_bal_column_color,
    ');
})();
\end{lstlisting}

\subsection{Datenbankmodell}
\label{app:Datenbankmodell}
ER-Modelle kann man auch direkt mit \LaTeX{} zeichnen, siehe \zB \url{http://www.texample.net/tikz/examples/entity-relationship-diagram/}.
\begin{figure}[htb]
\centering
\includegraphicsKeepAspectRatio{database.pdf}{1}
\caption{Datenbankmodell}
\end{figure}
\clearpage

\input{Anhang/AnhangEntwuerfe.tex}
\clearpage
\subsection{Screenshots der Anwendung}
\label{Screenshots}
\begin{figure}[htb]
\centering
\includegraphicsKeepAspectRatio{Frontend1.png}{0.914}
\caption{Ausspielen der Komponenten im Frontend}
\end{figure}
\begin{figure}[htb]
\centering
\includegraphicsKeepAspectRatio{Backend1.png}{0.95}
\caption{Pflegen der Komponenten im Backend}
\end{figure}
\clearpage

\input{Anhang/AnhangDoc.tex}
\clearpage
\input{Anhang/AnhangTest.tex}

\subsection{Klasse: ComparedNaturalModuleInformation}
\label{app:CNMI}
Kommentare und simple Getter/Setter werden nicht angezeigt.
\lstinputlisting[language=php, caption={Klasse: ComparedNaturalModuleInformation}]{Listings/cnmi.php}
\clearpage

\subsection{Klassendiagramm}
\label{app:Klassendiagramm}
Klassendiagramme und weitere \acs{UML}-Diagramme kann man auch direkt mit \LaTeX{} zeichnen, siehe \zB \url{http://metauml.sourceforge.net/old/class-diagram.html}.
\begin{figure}[htb]
\centering
\includegraphicsKeepAspectRatio{Klassendiagramm.pdf}{1}
\caption{Klassendiagramm}
\end{figure}
\clearpage

\input{Anhang/AnhangBenutzerDoku.tex}

