% !TEX root = ../Projektdokumentation.tex
\section{Entwurfsphase} 
\label{sec:Entwurfsphase}

\subsection{Zielplattform}
\label{sec:Zielplattform}

\subsubsection{TypeScript}
\label{sec:TypeScript}

Als Programmiersprache wurde TypeScript(\acs{TS}) ausgewählt\footnote{\Vgl \citet{TypeScript}.}. \acs{TS} ist ein Superset von JavaScript(\acs{JS}). Dies bedeutet, dass jeder JS-Code in \acs{TS} funktioniert, aber nicht jeder \acs{TS}-Code in JS funktioniert. \acs{TS} erweitert JS um Typen. Dies bedeutet, dass Variablen feste Datentypen(string, number, etc.) haben können. Dies hat den Vorteil, dass \acs{TS} beim kompilieren Fehler schmeißt, falls eine Variable einen Wert bekommt, welche sie nicht haben sollte. \acs{TS} wird zu JS kompiliert, da Browser aktuell nur JS und WebAssembly unterstützen. Dadurch werden vor allem große Projekte weniger fehleranfällig. Das Skeleton selber ist nicht so ein großes Projekt, aber es werden große Projekte damit entwickelt werden. Deswegen ist es sinnvoll, bereits mit \acs{TS} zu starten.

\subsubsection{Nuxt 3}
\label{sec:Nuxt 3}

Als Framework für TS wurde sich für Nuxt 3 entschieden. Nuxt 3 ist ein serverseitig gerendertes Javascript-Framework mit TypeScript Support. Nuxt 3 bietet viele Features, welche Webentwicklung einfacher machen. Dazu zählt beispielsweise reactivty. Dies bedeutet, dass das Framework automatisch Variablen in der Darstellung anpassen kann, wenn diese ihren Wert ändern. Viele andere Javascript-Frameworks werden clientseitig gerendert. Das bedeutet, dass der Nutzer das HTML erst rendern muss. Nuxt3 macht dies bereits auf dem Server, wodurch die Seite beim Nutzer schneller angezeigt wird. Dies wirkt sich auch vorteilhaft auf SEO aus.

\subsubsection{DaisyUI / Tailwind}
\label{sec:DaisyUI / Tailwind}

Als UI-Framework wurde DaisyUI genutzt. Für die Umsetzung des Skeletons war die Nutzung eines UI-Frameworks nicht wirklich nötig. Insgesamt musste während der Entwicklung des Skeleton nicht viel CSS / Styling genutzt werden. Das Framework liefert vorgestylte Komponenten und CSS-Klassen, welche die zukünftige Entwicklung beschleunigen sollen. DaisyUI basiert auf Tailwind, weswegen alle Tailwind Klassen auch in dem Projekt genutzt werden können.

\subsubsection{Docker}
\label{sec:Docker}

Um Typo3 besser testen zu können, wird Typo3 in einem Docker-Container ausgeführt. Docker ist eine Open-Source-Tool, welches es erlaubt, Software in abgekapselten Umgebungen/Containern laufen zu lassen. Ein Docker-Container ist Softwarepaket, was alle nötigen Werkzeuge enthält, um ein Programm laufen zu lassen. Docker ermöglicht es, verschiedenste Programme (andere Programmiersprachen, Systemanforderungen, Libraries, etc.) in unterschiedlichen Containern laufen zu lassen. Beispielsweise kann auf einem Rechner/Server in einem Docker-Container Typo3 laufen, während in einem anderen Nuxt läuft. Docker-Container können schnell hoch und runtergefahren werden, was testen von neuem Code, oder neustarten von Systemen sehr schnell macht. 

Bei Docker-Containern handelt es sich nicht um virtuelle Maschinen. Vielmehr sind Docker-Container virtuelle Betriebssysteme. Dadurch können Docker-Container mehr Ressourcen mit ihrem Host-System teilen. Deswegen verbrauchen sie insgesamt weniger Leistung als virtuelle Maschinen.

\subsubsection{Sonstiges}
\label{sec:Sonstiges}

Sonst wurden die klassischen Webtechnologien HTML und CSS genutzt.

Genauere Vergleiche zu anderen Frameworks / Technologien, werden im folgenden Kapitel analysiert.


\subsection{Architekturdesign}
\label{sec:Architekturdesign}

Anhand folgender Bewertungsmatrix, wurde sich für Nuxt 3 als JS-Framework entschieden:

\tabelle{Entscheidungsmatrix Nuxt 3}{tab:Entscheidungsmatrix Nuxt 3}{Nutzwert_Nuxt3.tex}

Als Optionen wurden alle Frameworks genommen, mit welchen \acs{BAL} bereits Erfahrungen gesammelt hat. Eigenentwicklungen wurden von Anfang an ausgeschlossen, da diese zu aufwendig sind. Andere Frameworks hätten eine zu lange Einarbeitung/Umschulung für potenziell zu wenig Nutzen gehabt.

Anhand folgender Bewertungsmatrix, wurde sich für DaisyUI als UI-Framework entschieden:

\tabelle{Entscheidungsmatrix DaisyUI}{tab:Entscheidungsmatrix DaisyUI}{Nutzwert_DaisyUI.tex}

Anhand folgender Bewertungsmatrix, wurde sich für TS als Programmiersprache entschieden:

\tabelle{Entscheidungsmatrix TS}{tab:Entscheidungsmatrix TS}{Nutzwert_TS.tex}

Beide Programmiersprachen werden von Nuxt 3 standardmäßg unterstützt. \acs{TS} benötigt mehr Zeilen Code und fügt einen weiteren Layer an Komplexität zu dem Projekt hinzu. Dafür werden Fehler früher entdeckt und das Projekt lässt sich leichter skalieren. Diese Faktoren werden von \acs{BAL} mehr wertgeschätzt, weswegen sich für \acs{TS} entschieden wurde.

Für das Typo3 musste keine Entscheidungsmatrix erstellt werden, da Typo3 von wichtigen Kunden vorgegeben wird. Es wird unabhängig vom Nutzen der Technologie ein Skeleton mit dieser Technologie benötigt. Typo3 hat unabhängig davon aber einige Vorteile, welche bereits erläutert wurden.

Navigation: Standardmäßig gibt die Headless-Extension von Typo3 keine Navigationsstruktur aus, weswegen noch Anpassungen am Template gemacht werden mussten. Es wurde sich entschieden, dass alle Seiten(und deren Unterseiten) in die Navigation ausgespielt werden, welche einer bestimmten Kategorie zugewiesen werden. Kategorien sind standardmäßig bei Seiten in Typo3 vorhanden und können mit kleinen Template-Anpassungen ausgespielt werden. Die Headless Extension bietet dafür Funktionalitäten.

Layout / Spalteneinteilung: Standardmäßig hat Typo3 keine Tools, um Komponenten in Spalten/Bereiche aufzuteilen. Es gibt viele Standardkonfigurationen, welche dies ermöglichen, aber ein komplett leeres Typo3 hat diese Funktionalität nicht. Deswegen wird eine eigene kleine Extension für Typo3 geschrieben, welche diese Funktionalität liefert.


\subsection{Entwurf der Benutzeroberfläche}
\label{sec:Benutzeroberflaeche} 

\paragraph{Frontend} 
\linebreak
Das Ziel der Benutzeroberfläche des Skeletons ist, dass Komponenten dargestellt werden und zwischen Seiten des Typo3 navigiert werden kann. Es muss noch keine gute Benutzeroberfläche erstellt werden, welche Kunden nützen können. Stattdessen muss es für Entwickler einfach und flexibel sein, eine gute Benutzeroberfläche für Kunden aus der Benutzeroberfläche des Skeletons, zu entwickeln. Dafür werden die Komponenten und die Navigation stilistisch sehr minimalistisch gehalten und primär ihre Logik erstellt. Die Navigation befindet sich klassisch am Anfang der Website und unter ihr wird der Content der Seite gerendert. Dies wurde auch hier angewandt.


\paragraph{Backend} 
\linebreak 
Die Benutzeroberfläche des Backends/CMS ist größtenteils durch Typo3 vorgegeben. Es werden lediglich die Komponenten erweitert, sodass gesteuert werden kann, wie viel Platz die Komponenten einnehmen. Dafür werden für die einzelnen Bildschirmgrößen(Desktop, Tablet, Mobile) Selects eingebaut, welche die Größen als Optionen haben.

\paragraph{Menüführung}  
\linebreak
Die Hauptnavigationspunkte, welche aus dem Typo3 ausgespielt werden, werden oben in der Menüleiste angezeigt. Ihre Kinder(Children/Unterseiten) werden ebenfalls ausgespielt, aber erst angezeigt, wenn durch die Navigation navigiert wird. Wenn einer der Hauptnavigationspunkte angeklickt wird, wird überprüft, ob dieser Children(Unterseiten) hat oder nicht. Wenn dies nicht der Fall ist, wird direkt die Seite von dem Hauptnavigationspunkt aufgerufen. Wenn dies der Fall ist, wird das Slidemenu aufgerufen. Das Slidemenu enthält sowohl den zuletzt angeklickten Navigationspunkt als auch alle Children von ihm. Es ist dazu da, um durch tiefere Navigationsebenen der Website zu steuern. Der Name des Slidemenu kommt von einer CSS Animation, welche das Menü in den Screen sliden lässt. 

Eine Darstellung der Menüführung findet sich im \Anhang{Screenshots des Menüs}. 

Zusätzlich gibt es einen Button, mit welchem zurück navigiert werden kann. Wenn der zuletzt angeklickte Navigationspunkt hier noch mal angeklickt wird, wird dessen Seite aufgerufen. Wenn ein Child angeklickt wird, wird wieder überprüft, ob es Children hat oder nicht. Falls nein, wird die Seite aufgerufen, falls ja wird das Child zum zuletzt angeklickten Navigationspunkt. 

\paragraph{rendern der verschiedenen Typo3-Inhaltselemente}
Im Typo3 können viele verschiedene Inhaltselemente gepflegt werden. Um diese darzustellen, wurde sich entschieden, für jedes Inhaltselement eine eigene Komponente zu erstellen. Für die Zuweisung der Inhaltselemente zu ihrer jeweiligen Nuxt 3 Komponente wurde eine weitere Nuxt 3 Komponente(TypoView.vue) erstellt. Diese analysiert/verarbeitet den Content der Seite und kümmert sich um die Zuweisungen.

\subsubsection{Typo3 Datenverarbeitung durch Nuxt / Flussdiagramm}
\label{sec:Typo3 Datenverarbeitung durch Nuxt / Flussdiagramm} 

Um die Typo3 Datenverarbeitung von Nuxt 3 darzustellen, wurde ein Flussdiagramm erstellt. siehe Abbildung \ref{FlussDiagramm} des Anhangs. Wie sich im Flussdiagramm erkennen lässt, werden alle Daten als erstes von einer Komponente([...slug].vue) ausgelesen. Diese Datei wertet teilweise die Daten aus, verschickt aber auch Teile der Daten weiter an andere Komponenten. 

Die Navigationsdaten im page Objekt werden an die Navigation.vue Komponente weitergeleitet. Die Navigation.vue Komponente besteht aus zwei weiteren Komponenten. Einmal der TopNavigation.vue und einmal der SlideMenu.vue. Alle drei Komponenten nutzen das navigation.ts Composable um miteinander zu kommunizieren und die Menüführung darzustellen.

Der Content wird an die Typoview.vue Komponente weitergeleitet. Dies ist die Komponente, welche sich um die Zuweisung der Typo3-Inhaltselemente zu den jeweiligen Nuxt 3 Komponenten kümmert. 

Die Breadcrumbs werden in die Breadcrumbs.vue Komponente weitergeleitet. Dort werden diese auch direkt dargestellt. 

Die Metadaten werden direkt in der [...slug].vue Komponente zu den Metainformationen der Website hinzugefügt.

\subsection{Maßnahmen zur Qualitätssicherung}
\label{sec:Qualitaetssicherung}

Um die SEO von dem Skeleton abschätzen zu können, kann Lighthouse von Google verwendet werden. Lighthouse analysiert die Seite bezüglich SEO und liefert abhängig davon einen Wert von 0 – 100. Falls es SEO Schwierigkeiten gibt, werden diese hier erläutert. Da noch nicht viel Content auf den Seiten gepflegt ist, kann es sein, dass es dadurch zu Abzügen kommt. Dies wird sich aber dann automatisch anpassen, wenn mit dem Skeleton eine vernünftige Website erstellt wird.

Wie gut der Code bezüglich Flexibilität und Leserlichkeit ist, lässt sich am besten durch Feedback von Arbeitskollegen erfassen. Da bereits Projekte mit dem Skeleton in Umsetzung sind und mehrere Arbeitskollegen des Prüflings mit dem Code arbeiten, wurde viel Feedback generiert. Dieses viel größtenteils positiv aus.

\subsection{Pflichtenheft}
\label{sec:Pflichtenheft}

Ein Pflichten- oder Lastenheft wurde für dieses Projekt nicht erstellt. 

