% !TEX root = ../Projektdokumentation.tex
\section{Fazit} 
\label{sec:Fazit}

\subsection{Soll-/Ist-Vergleich}
\label{sec:SollIstVergleich}

Im Groben wurden alle Projektziele erreicht. Es existiert jetzt eine Nuxt 3 Anwendung, welche sich mit einem Headless-Typo3 verbindet und dessen content ausspielt. Einige Projektphasen haben etwas mehr oder weniger Zeit gebraucht, es ist aber alles in einem zu erwarteten Rahmen geblieben(< +-2 Stunden).

\tabelle{Soll-/Ist-Vergleich}{tab:Vergleich}{Zeitnachher.tex}

Wie zu sehen ist, sind die Abänderungen, welche zur Zeitplanung in \ref{sec:Projektphasen} getroffen wurden, gut gewählt worden. Bei der ursprünglichen Dauer der Dokumentation wäre eine Differenz von 20 Stunden aufgetreten. Diese wäre potentiell am Ende des Projektes nicht mehr einzuholen gewesen. Die Implementierungsphase lies sich glücklicherweise schneller umsetzen als ursprünglich gedacht. Der Prüfungsbewerber konnte sich schnell in Nuxt 3 einarbeiten.

Die einzig große Abweichung vom Projektantrag war der Bearbeitungszeitraum.



\subsubsection{Abweichungen Bearbeitungszeitraum}
\label{sec:AbweichungenProjektantrag}
Der geplante Bearbeitungszeitraum wurde relativ schnell im Projekt nach hinten verschoben. Der Grund waren mehrere Covid-19 Erkrankungen im Betrieb. Der Prüfungsbewerber selber wurde ebenfalls während des geplanten Zeitraums für ein paar Tage krank. Dadurch konnte der Prüfungsbewerber weniger Zeit in die Projektarbeit stecken als geplant, weswegen das Ende der Bearbeitungszeit in den Dezember verschoben wurde. Ursprünglich war geplant, dass der Bearbeitungszeitraum am 20.09.2022 startet und am 15.11.2022 endet. Der ursprüngliche Bearbeitungszeitraum wurde glücklicherweise so bewusst gewählt, dass er bei Notfällen etwas verlängert werden kann. Der finale Abgabetermin war der 14.12.2022 und konnte eingehalten werden.

\subsection{Lessons Learned}
\label{sec:LessonsLearned}

Bei zukünftigen Entwicklungen wird sich der Prüfungsbewerber genauer die Anforderungen durchlesen. Ihm wurde zu spät bewusst, dass die Dokumentation einen größeren Umfang einnimmt als gedacht. Dadurch musste die Zeitplanung geändert werden. In dem Fall der Projektarbeit lief dies noch gut, aber es hätte auch zu großen Problemen führen können.

Nuxt 3 hat sich als sehr einfach und schnell zu lernendes Framework herausgestellt. Diese Aussage steht natürlich unter dem Hintergrund, dass der Prüfungsbewerber bereits Nuxt 2 Erfahrung hatte. Dadurch konnte glücklicherweise die Implementierungsphase schneller durchgesetzt werden, als ursprünglich gedacht. Das hat mit dazu beigetragen, dass der Zeitplan eingehalten werden konnte. Dies gekoppelt mit dem Feedback von den Arbeitskollegen, welche angefangen haben das Skeleton zu nutzen, macht \acs{BAL} zuversichtlich Nuxt 3 in Zukunft zu nutzen.

\subsection{Ausblick}
\label{sec:Ausblick}

Wie bereits mehrfach in der Dokumentation erwähnt wird das Skeleton wahrscheinlich noch einige Iterationen durchleben. Ein paar dieser Iterationen sind dem begrenzten Umfang der Projektarbeit geschuldet. Es sind viele Standardkomponenten und Funktionalitäten eines Typo3 noch nicht dargestellt. Ein Beispiel wäre der Formularbaukasten. Praktisch jede Website hat heutzutage ein Kontaktformular, es würde Sinn ergeben diese Funktionalität mit in das Skeleton zu integrieren. Andere sind dem begrenzten Wissen über Nuxt 3 geschuldet. Rückblickend wird sich bestimmt offenbaren, dass bestimmte Abschnitte des Programmcodes effizienter/besser hätten programmiert werden können. Einige Anforderungen an das Skeleton werden sich erst bei dem entwickeln von Projekten ergeben. Die ersten Projekte werden zum aktuellen Zeitpunkt mit dem Skeleton entwickelt. Dabei werden mit hoher Wahrscheinlichkeit neue Wünsche/Anforderungen an das Skeleton hervorgehen.
