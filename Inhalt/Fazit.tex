% !TEX root = ../Projektdokumentation.tex
\section{Fazit} 
\label{sec:Fazit}

\subsection{Soll-/Ist-Vergleich}
\label{sec:SollIstVergleich}

\begin{itemize}
	\item Wurde das Projektziel erreicht und wenn nein, warum nicht?
	\item Ist der Auftraggeber mit dem Projektergebnis zufrieden und wenn nein, warum nicht?
	\item Wurde die Projektplanung (Zeit, Kosten, Personal, Sachmittel) eingehalten oder haben sich Abweichungen ergeben und wenn ja, warum?
	\item Hinweis: Die Projektplanung muss nicht strikt eingehalten werden. Vielmehr sind Abweichungen sogar als normal anzusehen. Sie müssen nur vernünftig begründet werden (\zB durch Änderungen an den Anforderungen, unter-/überschätzter Aufwand).
\end{itemize}

Im Groben wurden alle Projektziele erreicht. Es existiert jetzt eine Nuxt 3 Anwendung, welche sich mit einem Headless-Typo3 verbindet und dessen content ausspielt. Das einzige, was leider nicht mehr in der Projektzeit umzusetzen war, ist die Dokumentation für Entwickler. Als temporärer Ersatz kann diese Dokumentation dienen. Einige Projektphasen haben etwas mehr oder weniger Zeit gebraucht, es ist aber alles in einem zu erwarteten Rahmen geblieben(< +-2 Stunden).

\tabelle{Soll-/Ist-Vergleich}{tab:Vergleich}{Zeitnachher.tex}


\subsection{Lessons Learned}
\label{sec:LessonsLearned}

\begin{itemize}
	\item Was hat der Prüfling bei der Durchführung des Projekts gelernt (\zB Zeitplanung, Vorteile der eingesetzten Frameworks, Änderungen der Anforderungen)?
\end{itemize}

Bei zukünftigen Entwicklungen wird sich der Prüfungsbewerber genauer die Anforderungen durchlesen. Ihm wurde zu spät bewusst, dass die Dokumentation einen größeren Umfang einnimmt als gedacht. Dadurch musste die Zeitplanung geändert werden. In dem Fall der Projektarbeit lief dies noch gut, aber es hätte auch zu großen Problemen führen können. Dass der Prüfungsbewerber etwas Pufferzeit eingeplant hatte, was den Abgabetermin anging, hat sich als sehr sehr vorteilhaft erwiesen. Ohne diese Pufferzeit, hätten die diversen Ausfälle im Betrieb und beim Prüfungsbewerber selber zu großen Problemen geführt.

Nuxt 3 hat sich als sehr einfach und schnell zu lernendes Framework herausgestellt. Diese Aussage steht natürlich unter dem Hintergrund, dass der Prüfungsbewerber bereits Nuxt 2 Erfahrung hatte. Dadurch konnte glücklicherweise die Implementierungsphase schneller durchgesetzt werden, als ursprünglich gedacht. Das hat mit dazu beigetragen, dass der Zeitplan eingehalten werden konnte. Dies gekoppelt mit dem Feedback von den Arbeitskollegen, welche angefangen haben das Skeleton zu nutzen, macht \acs{BAL} zuversichtlich Nuxt 3 in Zukunft zu nutzen.

\subsection{Ausblick}
\label{sec:Ausblick}

Wie bereits mehrfach in der Dokumentation erwähnt wird das Skeleton wahrscheinlich noch einige Iterationen durchleben. Ein paar dieser Iterationen sind dem begrenzten Umfang der Projektarbeit geschuldet. Es sind viele Standardkomponenten und Funktionalitäten eines Typo3 noch nicht dargestellt. Ein Beispiel wäre der Formularbaukasten. Praktisch jede Website hat heutzutage ein Kontaktformular, es würde Sinn ergeben diese Funktionalität mit in das Skeleton zu integrieren. Andere sind dem begrenzten Wissen über Nuxt 3 geschuldet. Rückblickend wird sich bestimmt offenbaren, dass bestimmte Abschnitte des Programmcodes effizienter/besser hätten programmiert werden können. Einige Anforderungen an das Skeleton werden sich erst bei dem entwickeln von Projekten ergeben. Die ersten Projekte werden zum aktuellen Zeitpunkt mit dem Skeleton entwickelt. Dabei werden mit hoher Wahrscheinlichkeit neue Wünsche/Anforderungen an das Skeleton hervorgehen.

\begin{itemize}
	\item Wie wird sich das Projekt in Zukunft weiterentwickeln (\zB geplante Erweiterungen)?
\end{itemize}
