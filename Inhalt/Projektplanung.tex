% !TEX root = ../Projektdokumentation.tex
\section{Projektplanung} 
\label{sec:Projektplanung}


\subsection{Projektphasen}
\label{sec:Projektphasen}

Die Bearbeitung der Projektarbeit inklusive Dokumentation fand vom 23.09.2022 - 11.12.2022 statt. Die Planungs- und Implementierungsphasen schließen im Oktober ab. Die Dokumentation wurde in den folgenden Monaten abgeschlossen.

Die im Projektantrag geplante Zeitplanung sah wie folgt aus:

Tabelle~\ref{tab:Zeitplanung} zeigt die alte grobe Zeitplanung des Projektes.
\tabelle{alte grobe Zeitplanung}{tab:Zeitplanung}{ZeitplanungKurz}


Es wurden aber Änderungen an der Zeitplanung vorgenommen, die neue Zeitplanung sieht wie folgt aus:

Tabelle~\ref{tab:Zeitplanung} zeigt die neue grobe Zeitplanung des Projektes.
\tabelle{neue grobe Zeitplanung}{tab:Zeitplanung}{ZeitplanungKurzNeu}

Warum vom Projektantrag abgewichen wurden, wird in Kapitel \ref{sec:AbweichungenProjektantrag} Abweichungen vom Projektantrag erklärt.

\subsubsection{Implementierungsphase Zeitplanung}
\label{sec:Implementierungsphase Zeitplanung}

Die Implementierungsphase nimmt den mit Abstand größten Teil der Projektarbeit ein. Dadurch ist es sinnvoll hier eine detailierte Zeitplanung zu erstellen.

Tabelle~\ref{tab:Zeitplanung Implementierungsphase} zeigt die Zeitplanung der Implementierungsphase.
\tabelle{Zeitplanung Implementierungsphase}{tab:Zeitplanung Implementierungsphase}{ZeitplanungImplementierungsphase}

Wie zu sehen ist, wird geplant, dass die Nuxt 3 Entwicklungen 80\% des Arbeitsumfangs ausmachen werden. Dies liegt daran, dass ein großer Teil der Arbeitsaufwände durch das installieren der Headless Extension bereits erledigt werden. Der Prüfungsbewerber hofft zusätzlich, dass er die Erweiterungen des Typo3 teilweise aus alten Projekten übernehmen kann. Ob diese mit der neusten Typo3 Version funktionieren, wird sich im Verlaufe des Projekts zeigen. Falls Code aus alten Projekten recycled wird, wird dies im Text erkennbar gemacht. Dementsprechend wird sich Kapitel \ref{sec:Implementierungsphase} Implementierungsphase auch primär auf die Nuxt 3 Entwicklungen fokussieren.

\subsection{Abweichungen vom Projektantrag}
\label{sec:AbweichungenProjektantrag}

An dem Projekt als solchem hat sich nichts geändert. Es wurde das gleiche Stück Software erstellt, wie es im Projektantrag beschrieben wurde. Fehlerhafte Planung hat es leider notwendig gemacht, dass Anpassungen an der Zeitplanung getroffen werden mussten.


\subsubsection{Zeitplanung}
\label{sec:AbweichungenProjektantrag}

Kurz nach einreichen des Projektantrages, wurde dem Prüfungsbewerber relativ schnell klar, dass die Zeitplanung so nicht funktionieren würde. Er hatte im Projektantrag nur 5 Stunden für die Dokumentation des Projektes eingeplant. Das entspricht weniger als 9 Minuten pro Seite(300 Minuten / 35 Seiten). In so einem kurzem Zeitraum lässt sich keine angemessene Dokumentation erstellen. Der Prüfungsbewerber ist ursprünglich davon ausgegangen, dass der Projektantrag, aufgrund dieser Zeit, abgelehnt werden würde. Dies ist nicht geschehen. Da er aber unmöglich diese Dokumentation in 5 Stunden schreiben konnte, wurden die Zeiten der Projektphasen abgeändert. Dafür wurden 15 Stunden aus der Implementierungsphase entnommen. Diese Phase hat mit großem Abstand die meisten Stunden. Selbst mit Entnahme der Stunden, hatte diese Phase immer noch die Hälfte der gesamten Projektzeit. Hier wurden bei der Planung des Projekts die Zeiten auch am großzügigsten geschätzt. 5 weitere Stunden wurden aus der Analysephase entnommen. Da ein guter Teil der Arbeit in dieser Phase bereits beim einreichen des Projektantrages erledigt wurde. So hatte der Prüfungsbewerber nun 25 Stunden Zeit die Dokumentation zu schreiben.

Ob die neue Planung funktioniert hat, oder ob es während des Projektes noch zu anderen Problemen kam, wird im Fazit analysiert und besprochen.

\subsection{Ressourcenplanung}
\label{sec:Ressourcenplanung}

Für die Umsetzung des Projektes wurden folgende Ressourcen benötigt:
\begin{itemize}
	\item Computer
	\item Internetzugang
	\item Büroraum
	\item Quellcode von Typo3 und Nuxt 3 (kostenlos übers Internet verfügbar)
	\item Personelle Ressourcen: senior Developer für Rückfragen
\end{itemize}

Zusätzlich wurde eine LaTeX-Vorlage zum erstellen der Dokumentation verwendet.  Die LaTeX-Vorlage\footnote{\Vgl https://it-berufe-podcast.de/vorlage-fuer-die-projektdokumentation/} wurde von Stefan Macke\footnote{\Vgl Blog des Autors: http://fachinformatiker-anwendungsentwicklung.net, Twitter: @StefanMacke.} entwickelt. Die Vorlage steht unter der Creative Commons Namensnennung Lizenz. Die Vorlage selbst hat keinen Inhalt. Sie dient als eine grobe Strukturierung der Dokumentation.

\subsection{Entwicklungsprozess}
\label{sec:Entwicklungsprozess}

Bei der Entwicklung des Projektes wurde ein Wasserfall Entwicklungsprozess benutzt. Dies bedeutet dass die Projektphasen linear und ohne Rückschritte nacheinander abgearbeitet wurden. Die vordefinierten Projektphasen ergaben sich aus dem Projektantrag.
