% !TEX root = ../Projektdokumentation.tex
\section{Projektplanung} 
\label{sec:Projektplanung}


\subsection{Projektphasen}
\label{sec:Projektphasen}

\begin{itemize}
	\item Verfeinerung der Zeitplanung, die bereits im Projektantrag vorgestellt wurde.
\end{itemize}

Die Bearbeitung der Projektarbeit fand vom 23.09.2022 - 07.10.2022 statt. Die Tagesarbeitszeit war standardmäßig 8 Stunden. Gearbeitet wurde an allen Tagen, bis auf den 28.09.2022. Der Prüfungsbewerber hatte an dem Tag Berufsschule. 

Die im Projektantrag geplante Zeitplanung sah wie folgt aus:

\newpage

Tabelle~\ref{tab:Zeitplanung} zeigt die grobe Zeitplanung des Projektes.
\tabelle{grobe Zeitplanung}{tab:Zeitplanung}{ZeitplanungKurz}

Es wurde ebenfalls eine detaillierte Zeitplanung erstellt. In ihr werden die jeweiligen Projektphasen in weitere kleinere Teilabschnitte unterteilt. Dies dient sowohl dem besseren Verständnis des Projektes und liefert ebenfalls einen Handlungsfaden um das Projekt abzuarbeiten. Die Teilabschnitte der Projektphasen haben ebenfalls eine geplante Zeit, welche zusammen addiert die Zeit der jeweiligen Projektphase ergibt.

\newpage

Tabelle~\ref{tab:ZeitplanungKomplett} zeigt die detaillierte Zeitplanung des Projektes.
\tabelle{detaillierte Zeitplanung}{tab:ZeitplanungKomplett}{ZeitplanungKomplett}

\newpage

\subsection{Abweichungen vom Projektantrag}
\label{sec:AbweichungenProjektantrag}

\begin{itemize}
	\item Sollte es Abweichungen zum Projektantrag geben (\zB Zeitplanung, Inhalt des Projekts, neue Anforderungen), müssen diese explizit aufgeführt und begründet werden.
\end{itemize}


\subsection{Ressourcenplanung}
\label{sec:Ressourcenplanung}

Für die Umsetzung des Projektes wurden folgende Ressourcen benötigt:
\begin{itemize}
	\item Computer
	\item Internetzugang
	\item Büroraum
	\item Quellcode von Typo3 und Nuxt3 (kostenlos übers Internet verfügbar)
	\item Personelle Ressourcen: senior Developer für Rückfragen
\end{itemize}

\subsection{Entwicklungsprozess}
\label{sec:Entwicklungsprozess}

Bei der Entwicklung des Projektes wurde ein Wasserfall Entwicklungsprozess benutzt. Dies bedeutet dass die Projektphasen linear und ohne Rückschritte nacheinander abgearbeitet wurden. Die vordefinierten Projektphasen ergaben sich aus dem Projektantrag.
