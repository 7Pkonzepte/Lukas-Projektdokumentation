% !TEX root = ../Projektdokumentation.tex
\section{Dokumentation}
\label{sec:Dokumentation}

Wie in den Unterpunkten vom Anhang \ref{app:Doc} Entwicklerdokumentation zu finden ist, wurde für dieses Projekt eine Entwicklerdokumentation angelegt. Es wurde keine Kundendokumentation angelegt. Dies hat den Grund, dass die Abnehmer des Projekts Entwickler der bits \& likes GmbH sind. Aus den in der Zukunft erstellten Projekten sollten aber Kundendokumentationen erstellt werden. Da diese aber komplett unterschiedliche Designs und neue Komponenten haben werden, ergibt es kein Sinn in dem Kontext des Projektes eine zu erstellen. Der Großteil müsste umgeschrieben werden, wenn Projekte mit dem Skeleton umgesetzt werden. Für die Entwickler wurde dokumentiert, wie ein Typo3 in Docker gestartet werden kann und wie das Typo3 mit Nuxt 3 interagiert. Die Entwickler sind damit dann in der Lage das Projekt zu erweitern und komplexere Designs umzusetzen. Was in der Dokumentation noch fehlt, ist eine Beschreibung, wie man das Typo3 selber erweitert. Für Extensions hat Typo3 selbst eine gute Entwicklerdokumentation. Diese hier wird in der Doku verlinkt. Selber alles aufzulisten würde den Rahmen der Entwicklerdokumentation sprengen. Um die Extension-Entwicklung von Typo3 vollständig darzustellen, bedarf es mehr weit als 10 Seiten.