% !TEX root = ../Projektdokumentation.tex
\section{Implementierungsphase} 
\label{sec:Implementierungsphase}

\subsection{Docker-Setup}
\label{sec:Docker-Setup}

Als Image für das aktuellste Typo3 Docker-Setup wurde das Image von Martin Helmich genutzt (https://github.com/martin-helmich/docker-typo3). Ein Docker-Image ist eine Datei, welche aus Anweisungen besteht, welche vollständige und ausführbare Version einer Anwendung erstellt. Diese Anweisungen werden in einer docker-compose.yml Daten gespeichert und dann mit dem Befehl docker-compose up ausgeführt. Nach dem Ausführen des Befehls, wird eine Typo3 Instanz erstellt und gestartet. Abhängig von dem angegebenen Port, kann dann das Typo3 über Localhost aufgerufen werden. Nach dem ausführen der Anweisungen, wird dort der Installations-Screen(siehe Abbildung) von Typo3 angezeigt. Nach dem Installieren, kann ausgewählt werden, ob das Typo3 mit einer leeren Startseite gestartet werden soll, oder weitere Konfigurationen installiert/ausgeführt werden sollen. Um das Skeleton möglichst minimal zu halten, wird das Typo3 mit einer leeren Startseite initialisiert.

\subsection{Ausspielen der Typo3 Daten}
\label{sec:Ausspielen der Typo3 Daten}

Damit die Typo3 Daten im JSON-Format ausgespielt werden, wird die Headless-Extension \linebreak (https://extensions.typo3.org/extension/headless) installiert. Danach muss die Extension zum Template hinzugefügt werden, damit Typo3 weiß, dass es das Template dementsprechend anpassen muss. Typo3 spielt nun die Daten im JSON-Format aus. Das JSON muss noch angepasst werden, damit eine Navigationsstruktur damit abgebildet werden kann. Dafür werden folgende Erweiterungen am Template vorgenommen.

\colorlet{punct}{red!60!black}
\definecolor{background}{HTML}{EEEEEE}
\definecolor{delim}{RGB}{20,105,176}
\colorlet{numb}{magenta!60!black}

\lstdefinelanguage{json}{
    basicstyle=\normalfont\ttfamily,
    numbers=left,
    numberstyle=\scriptsize,
    stepnumber=1,
    numbersep=8pt,
    showstringspaces=false,
    breaklines=true,
    frame=lines,
    backgroundcolor=\color{background},
    literate=
     *{0}{{{\color{numb}0}}}{1}
      {1}{{{\color{numb}1}}}{1}
      {2}{{{\color{numb}2}}}{1}
      {3}{{{\color{numb}3}}}{1}
      {4}{{{\color{numb}4}}}{1}
      {5}{{{\color{numb}5}}}{1}
      {6}{{{\color{numb}6}}}{1}
      {7}{{{\color{numb}7}}}{1}
      {8}{{{\color{numb}8}}}{1}
      {9}{{{\color{numb}9}}}{1}
      {:}{{{\color{punct}{:}}}}{1}
      {,}{{{\color{punct}{,}}}}{1}
      {\{}{{{\color{delim}{\{}}}}{1}
      {\}}{{{\color{delim}{\}}}}}{1}
      {[}{{{\color{delim}{[}}}}{1}
      {]}{{{\color{delim}{]}}}}{1},
}
\begin{lstlisting}[language=json,firstnumber=1]
lib.page = JSON
lib.page {
  fields {
    navigation {
      fields {
        main {
          dataProcessing {
            10 = FriendsOfTYPO3\Headless\DataProcessing\MenuProcessor
            10 {              
				       special = categories
				       special.value = 1
				       levels = 7
				       as = main
				       expandAll = 1
				       includeSpacer = 1
				       titleField = nav_title // title	
              	dataProcessing {
                	10 = FriendsOfTYPO3\Headless\DataProcessing\FilesProcessor
                	10 {
                  		references.fieldName = media
                  		as = media
                	}
              	}
                 overwriteMenuLevelConfig {
                  stdWrap.cObject {
                    100 = TEXT
                    100.field = uid
                    100.wrap = ,"uid":|
                  }
                }   
            }
          }
        }
      }      
    }
  }
}
\end{lstlisting}
Durch diese Anpassungen erhält das page Objekt die property navigation. Die Eigenschaft navigation hat wiederrum die Eigenschaft main. Mit der Hilfe des Menuprocessor von der Headless Extension, werden in die main Eigenschaft alle Seiten + Unterseiten geladen, welche der 1. Kategorie zugewiesen werden (special = categories \& special value = 1). Falls Bilder für die Seiten gepflegt sind, werden diese ebenfalls durch den FilesProcessor ausgespielt. Diese werde in dem Skeleton aber nicht verwendet. Dies dient nur zum Nutzen von potentiellen zukünftigen Projekten. Die Anpassungen können sowohl in der Skeleton-Extension stattfinden, als auch im Typo3 Backend.

\subsection{Skeleton-Extension / Layout}
\label{sec:Skeleton-Extension / Layout}
Typo3 ermöglicht es Entwicklern das Typo3 mit eigenen Extensions zu erweitern. Diese Extensions müssen einer bestimmten Struktur folgen, welche von Typo3 vorgegeben ist. Die genaue Form lässt sich in der Typo3 Dokumentation finden. \newline https://docs.typo3.org/m/typo3/reference-coreapi/main/en-us/ExtensionArchitecture/Index.html
\paragraph{Datenmodell:}

Alle Komponenten, die gepflegt werden können, sollen für Desktop, Tablet und Mobile verschiedene Breiten gepflegt bekommen können. Ein Text soll beispielsweise nur ein drittel der Bildschirmbreite auf Desktop haben, aber 50\% auf einem Tablet und die volle Breite auf einem Mobiltelefon. Dafür erhalten alle Elemente drei Felder in der Datenbank:
\begin{lstlisting}[language=json,firstnumber=1]
CREATE TABLE tt_content (
        tx_responsive_mobile int(11) DEFAULT '100' NOT NULL,
        tx_responsive_tablet int(11) DEFAULT '100' NOT NULL,
        tx_responsive_desktop int(11) DEFAULT '100' NOT NULL,
);
\end{lstlisting}
Anpassungen an bestehende Tabellen in der Datenbank werden in der ext\_tables.sql Datei vorgenommen. Diese Datei wird automatisch beim installieren von Extensions ausgelesen und die Anpassungen durchgeführt. Felder, welche als Input für Inhaltselemente dienen, werden in Typo3 in der tt\_content Tabelle gespeichert. Diese Felder müssen noch mit angemessenen Auswahlmöglichkeiten versehen werden und dann allen Elementen zugewiesen werden.
\paragraph{Backend:} Die Inputfelder werden als Selects dargestellt. Die Auswahlmöglichkeiten werden im Code vorgegeben.
\begin{lstlisting}[language=json,firstnumber=1]
"tx_responsive_mobile" => Array (
        "exclude" => 1,
        "label" => 'Mobile',
        "config" => Array (
            'type' => 'select',
            'renderType' => 'selectSingle',
            'items' => [
                ['25%','25'],
                ['33%','33'],
                ['50%','50'],
                ['66%','66'],
                ['75%','75'],
                ['100%','100']
            ],
            'default' => '100',
            'size' => 1,
            'maxitems' => 1,
        )
    ),
\end{lstlisting}
Analog zum oberen Code, werden für Tablet und Desktop ebenfalls Konfigurationen vorgenommen und im Array \$tempColumns gespeichert. Welche Eigenschaften Inputfelder brauchen und welche Optionen es bei den Inputfeldern gibt, lässt sich in der Typo3 Dokumentation(https://docs.typo3.org/m/typo3/\newline reference-tca/main/en-us/Columns/Index.html) nachlesen. Um die Inputfelder nun allen Elementen hinzuzufügen, wird die ExtensionManagementUtility Klasse aus dem Typo3 Core genutzt.
\begin{lstlisting}[language=json,firstnumber=1]
\TYPO3\CMS\Core\Utility\ExtensionManagementUtility::addTCAcolumns("tt_content",$tempColumns,1);
\TYPO3\CMS\Core\Utility\ExtensionManagementUtility::addToAllTCAtypes('tt_content','--div--;Responsive,tx_responsive_mobile,tx_responsive_tablet,tx_responsive_desktop','','after:addToAllTCAtypes');
\end{lstlisting}

Die erste Zeile Code speichert die Inputfelder ab, die zweite Zeile Code, fügt sie allen Inhaltselementen hinzu. --div--;Responsive sorgt dafür, dass diese in einem eigenen Tab mit dem Namen Responsive dargestellt werden. Danach wird aufgelistet werden Inputfelder hinzugefügt werden.

\paragraph{Ausspielen ans Frontend:}

Damit die gepflegten Daten auch ans Frontend ausgespielt werden, muss das Template von Typo3 um diese erweitert werden. Extensions haben dafür im Pfad \newline Configuration->TypoScript die Datei setup.typoscript. Hier können Anpassungen ans Template gepflegt werden, diese Anpassungen werden erst aktiv, wenn man die Extension dem Typo3 Template zuweist.

\begin{lstlisting}[language=json,firstnumber=1]
lib.appearance {
    fields {
        responsive_desktop = TEXT
        responsive_desktop {
            field = tx_responsive_desktop
        }
        responsive_tablet = TEXT ......
    }
}
\end{lstlisting}

Die gepflegten Daten werden nun in der Eigenschaft appearance ausgespielt und können von Nuxt3 ausgelesen werden.

\paragraph{Sections:}
Um die Elemente auf einer Seite in Bereiche auszuteilen, wurde ein komplett neues Inhaltselement erstellt. Dieses wurde als Section bezeichnet. Sections können Hintergrundfarben haben, wodurch Abschnitte im Frontend farblich voneinander getrennt dargestellt werden können. Dafür wurde ebenfalls in der tt\_content Tabelle ein Feld angelegt. In dem Fall mit dem Namen tx\_bal\_column\_color. Damit das neue Inhaltselement in Typo3 genutzt werden kann, muss die tt\_content.php Datei um folgenden Code erweitert werden. \Anhang{tt_content.php} \newline Die Datei lässt sich im Pfad Configuration->TCA->Overrides der Extension finden.

Damit das neue Element auch in der Auswahl von Elementen angezeigt wird, muss das Element noch in der ext\_localconf.php Datei im Root Verzeichnes der Extension hinzugefügt werden.\newline \Anhang{ext_localconf.php} \newline Übersetzungen für den Namen und die Beschreibung der Section wurden in einer .xlf Datei gespflegt.


\subsection{Setup der Implementierung der Benutzeroberfläche}
\label{sec:Setup der ImplementierungBenutzeroberflaeche}

\begin{itemize}
	\item Beschreibung der Implementierung der Benutzeroberfläche, falls dies separat zur Implementierung der Geschäftslogik erfolgt (\zB bei \acs{HTML}-Oberflächen und Stylesheets).
	\item \Ggfs Beschreibung des Corporate Designs und dessen Umsetzung in der Anwendung.
	\item Screenshots der Anwendung
\end{itemize}

\subsubsection{Nuxt3 aufsetzen um DaisyUI installieren}
\label{sec:Nuxt3 aufsetzen um DaisyUI installieren}
Um Nuxt 3 nutzen zu können, muss eine aktuelle Version von Node.js auf der Betriebssystem installiert werden. Zum installieren von Nuxt 3 und DaisyUI wurde der Node Package Manager genutzt. Um das neue Nuxt 3 Projekt aufzusetzen, wurde der Befehl npm nuxi init bal-skeleton ausgeführt. Dies erstellt ein neues Nuxt 3 Projekt mit dem Namen bal-skeleton in dem Verzeichnis, indem es ausgeführt wurde. Also Software für version-control wurde Github genutzt. Das Nuxt 3 Projekt kann nun, im entsprechenden Verzeichnis, durch den Befehl npm run dev gestartet werden. Im Browser wird nun auf localhost:3000 die Website dargestellt. Änderungen am Code werden direkt angewandt und im Browser angezeigt.

\subsection{Navigation der Implementierung der Benutzeroberfläche}
\label{sec:Navigation der der ImplementierungBenutzeroberflaeche}

\subsubsection{Dynamische Seiten}
\label{sec:Dynamische Seiten}
Websiten haben normalerweise nicht nur eine Startseite, sondern sie haben meisten auch mehrere Unterseiten. Damit Nuxt 3 weiß, welche Dateien als Seiten fungieren werden diese Dateien in ein bestimmtes Verzeichnis gelegt. Alle Seiten der Nuxt 3 Anwendungen befinden sich im pages Verzeichnis. Da Nuxt 3 und wir als Programmierer nicht wissen, wie der Seitenbaum von Typo3 aussieht und dieser sich ständig ändert, können die einzelnen Seiten nicht per Hand eingepflegt werden. Damit Nuxt3 trotzdem weiß welcher Code bei den unterschiedlichen Seiten ausgeführt werden soll, gibt es eine Catch-all Route. Eine Datei/Seite die den Namen [...slug].vue erhält, fungiert als Catch-all Route. Dies bedeutet, dass alle Seiten die nicht gepflegt sind, durch diese Datei verarbeitet werden. Wenn aber beispielsweise eine Shop-Seite angelegt werde soll, die ihre Daten durch eine E-Commerce Platform und nicht Typo3 bekommt. Dann kann im pages Ordner eine shop.vue Datei hinterlegt werden, welche dann für den Pfad /shop die Daten darstellt. Analog kann dies für alle möglichen Seiten geschehen. Das Skeleton bearbeitet also standardmäßig jeden Seitenaufruf gleich.

\subsubsection{Verarbeitung der Typo3 Daten}
\label{sec:Verarbeitung der Typo3 Daten}

In der nuxt.config.ts Datei können generelle Konfigurationen am Nuxt3 vorgenommen werden. In ihr wurde die URL des Typo3 gespeichert. Dies hat den Vorteil, dass die URL in der Anwendung als Variable genutzt werden kann. Das heißt, dass bei späteren Entwicklungen lediglich die Variable einen anderen Wert bekommen muss und Nuxt automatisch alle Daten von einem anderen Typo3 bezieht.

\begin{lstlisting}[language=json,firstnumber=1]
export default defineNuxtConfig({
    runtimeConfig: {
        public: {
            typo3: 'http://localhost'
        }
    },
})
\end{lstlisting}

Die Variable wird darauf hin in der [...slug].vue Datei genutzt um die Daten vom Typo3 zu laden. Dies geschieht im <script setup> Tag. Nuxt 3 Dateien können sowohl einen Script Tag haben, welcher Clientseitig ausgeführt wird, als auch ein Script Tag, welcher Serverseitig ausgeführt wird(hat den Zusatz setup). Da die Ladezeit minimal gehalten werden soll, werden die Typo3 Daten hier Serverseitig geladen. Zum Laden der Daten wird die Nuxt 3 eigene useAsyncData() Funktion genutzt. Mit ihr können Daten asynchron geladen werden. Zusätzlich hat sie viele Optionen um die Anfrage anzupassen. Eine der Optionen ist es den Cache auszuschalten. Für Entwicklungszwecke wird der Cache hier noch ausgeschaltet. Damit Änderungen im Typo3 direkt sichtbar werden. Wenn das Projekt in Produktion geht, sollte die Zeile {initialCache: false} entfernt werden. Dadurch Speichert Nuxt 3 die Daten im Cache und die Seite wird schnell geladen. Je nach Website kann programmiert werden, dass sich der Cache häufiger oder weniger häufiger leert.

\begin{lstlisting}[language=json,firstnumber=1]
  const runtimeConfig = useRuntimeConfig()
  const route = useRoute();
  const { data } = await useAsyncData(
      "pageData",
    () => $fetch(runtimeConfig.typo3 + route.fullPath),
    {initialCache: false}
  )
\end{lstlisting}

Im route Objekt ist der Pfad der aktuellen Website gespeichert. route.fullPath würde bei der Seite beispielseite.de/beispie1 dem Wert beispie1 entsprechen. Danach geht eine Anfrage an das Typo3 auf dem Pfad /beispiel. In diesem Fall wäre das http://localhost/beispiel. Das bedeutet, dass bei jedem Seitenaufruf, der Nuxt 3-Client das Typo3 anspricht. Wenn das Typo3 eine Seite unter dem Pfad hat, wird es die dementsprechenden Daten ausspielen. Normalerweise sollte dies der Fall sein, da die Elemente aus der Navigation alles Seiten aus dem Typo3 sind. Falls eine Seite angesteuert wird, die nicht im Typo3 existiert, wirft dieses einen 404 Fehler zurück. Das wurde hier in dem Projekt nicht beachtet. Sollte aber in zukünftigen Versionen des Projekts implementiert werden.

Die Daten aus dem Typo3(hier das data Objekt) wurden danach weiter verarbeitet. Der erste Schritt der Datenverarbeitung war es die Daten in Kategorien zu zerlegen. Dafür wurden vier Kategorien/Variablen angelegt. Die Variable breadcrumbs bekam die Daten bezüglich der Breadcrumbs der Seite. Die Variable Content bekam alle Daten bezüglich der Inhaltselemente auf der bestimmten Seite. Die Variable mainNavigation bekam alle Daten der Hauptnavigation. Als letztes bekam die Variable metaData alle Metadaten der Website. Die Breadcrumbs wurden bereits automatisch durch die Headless Extension generiert und sind im Datenobjekt vorhanden. Das gleiche gilt für die Metadaten und dem Großteil der Inhaltselemente. Die Navigationsdaten finden sich in der Eigenschaft main, welche zum Objekt navigation gehört, welches wiederum zum Objekt page gehört. Diese Logik wurde in \ref{sec:Ausspielen der Typo3 Daten} angelegt.

\begin{lstlisting}[language=json,firstnumber=1]
  let breadcrumbs = data.value.breadcrumbs;
  let content = data.value.content.colPos0;
  let mainNavigation = data.value.page.navigation.main;
  let metaData = data.value.meta;
\end{lstlisting}

Mit der Nuxt3 Funktion useHead() können die Metadaten an die Website weiter gegeben werden. Für die Weiterverarbeitung der Breadcrumbs, die Navigation und den Content wurden eigene Komponenten erstellt. Diese erhalten die Variablen als Eigenschaften/Properties/Props zugewiesen. Da Nuxt3 einen automatischen Import von Komponenten hat, müssen diese lediglich im Template gepflegt werden und nicht zusätzlich importiert werden. Dafür wurden die Komponenten im components Ordner gepflegt, damit Nuxt3 diese automatisch importieren kann. Den vollständigen Code der [...slug].vue finden Sie unter \Anhang{[...slug].vue}.

\subsubsection{Setup Navigation}
\label{sec:Setup Navigation}

Die Logik der Navigation wurde in eine Nuxt3 Composable ausgelagert. Die Navigation muss mehrere Aufgaben erfüllen. Den gesamten Code des Composables finden Sie unter \Anhang{navigation.ts}.

\begin{itemize}
\item öffnen/schließen des Slidemenu
\item anzeigen von allen Seiten
\item navigieren auf Seiten
\end{itemize}

Für das öffnen und schließen des Slidemenu wurden im Composable die Funktionen open() und close() erstellt. Der State, ob das Slidemenu geöffnet oder geschlossen ist, wird über die Variable navigationOpened gesteuert. Dafür wird die neue Nuxt3 eigene Funktion useState() genommen. Mit ihr lässt sich sehr einfach der State über die gesamt Anwendung verwalten. Der useState Funktion wird dabei ein Key übergeben, mit dem überall in der Anwendung auf den State zugegriffen werden kann. Zusätzlich kann der Wert des States optional initialisiert werden. Hier wird natürlich der Initialwert auf false gesetzt. Die Navigation soll sich erst öffnen, wenn der Nutzer der Website das möchte.
Analog wurde jeweils ein State für die gesamte Navigation(fullNavList), die aktuelle im Slidemenu angezeigte Navigation(currentNavList) und dem Parent der aktuell angezeigten Navigation(previousNavItem) erstellt.

\subsubsection{Aufteilung Navigation}
\label{sec:Aufteilung Navigation}

Die Navigation wurde in zwei Teile aufgeteilt, der erste Teil sind die Navigationspunkte, welche in der Navigationsbar der Website angezeigt werden. Sie sind die Seiten, welche im Typo3 die Kategorie 'main' zugewiesen bekommen haben. Der zweite Teil ist das Slidemenu. Das Slidemenu wird geöffnet, wenn ein Element in der Navigationsbar angeklickt wird und es Unterseiten(Children) hat. Im Slidemenu wird die currentNavList dargestellt. Initial sind dies die Unterseiten, des angeklickten Navigationselements. Die Navigationsbar wird in der Komponente TopNavigation.vue dargestellt und das Slidemenu in der Komponente SlideMenu.vue.

\subsubsection{TopNavigation.vue}
\label{sec:TopNavigation.vue}

TopNavigation.vue ist eine sehr einfache Komponente. Sie nutzt das navigation Composable um auf den State fullNavList zuzugreifen. Mit v-for, einer vue/nuxt internen Funktionalität, lässt sich im Template durch die Elemente durchiterieren. Mit @click lassen sich Funktionen aufrufen, wenn ein Element angeklickt wird. In diesem Fall ist es die open() Funktion des composables.
\begin{lstlisting}[language=json,firstnumber=1]
<div v-for="navItem in navigation.fullNavList.value" 
:key="navItem.uid" @click="navigation.open(navItem)">
	{{navItem.title}}	</div>
\end{lstlisting}
Es wurde noch einfaches Styling mit DaisyUI hinzugefügt, dies hat aber keinen Einfluss auf die Logik der Komponente und wird in zukünftigen Projekten wahrscheinlich überschrieben werden. Deswegen wird es hier nicht mit angezeigt.

\subsubsection{open() Funktion}
\label{sec:open() Funktion}

Die open() Funktion besteht aus einem if else Statement. Den Programmcode finden sie im Navigation Composable \Anhang{navigation.ts} (const open = ...). Wenn das Navigationselement(navItem), welches als Parameter angegeben wurde, keine Unterseiten hat, dann wird die Seite des navItems aufgerufen. Dafür wird der Nuxt-Router genutzt. Mit ihm kann programmatisch die Route geändert werden. Falls das Navigationselement Unterseiten hat, dann wird die aktuell angezeigt Navigation gleich den Unterseiten gesetzt. Als Parent der Navigation wird das Navigationsitem gesetzt und das Slidemenu, was diese Informationen darstellt, wird geöffnet.

\subsubsection{Slidemenu.vue}
\label{sec:Slidemenu.vue}

Den kompletten Code des Slidemenu finden Sie im Anhang unter \Anhang{SlideMenu.vue}.
Das Slidemenu zeigt die aktuelle Navigation an, welche aus Unterseiten eines anderen Navigationselement besteht. Dafür nutzt sie die gleiche Logik wie die Navigationsbar. Es wird genauso durch alle Elemente von currentNavList iteriert, wie es bei bei der Navigationsbar für fullNavList passiert ist. Der Unterschied liegt in der Funktion, welche bei @click hinterlegt wird. Es wird die Funktion navigate() aufgerufen, welche im Navigation Composable hinterlegt ist. Die Funktion navigate() ähnelt der Funktion open() sehr. Beide nehmen ein Navigationselement als Parameter. Beide Funktionen haben die gleichen if else Bedingungen. Der Unterschied liegt in dem Code, welcher nach der jeweiligen Bedingung ausgeführt wird. Wenn das Navigationselement keine Unterseiten hat, dann wird wie bei open() zur Seite navigiert. Zusätzlich wird aber auch das Slidemenu mit der close() Funktion geschlossen. Theoretisch könnte hier auch der State von navigationOpened direkt auf false gesetzt werden. Falls sich die close() Logik aber erweitert, muss dies dann nicht an meheren Orten angepasst werden. Falls das Element Unterseiten hat, wird die aktuelle angezeige Navigation den Unterseiten gleichgesetzt. 

\subsection{Implementierung des Contents der Benutzeroberfläche}
\label{sec:Implementierung des Contents der ImplementierungBenutzeroberflaeche}

\subsubsection{TypoView.vue}
\label{sec:TypoView.vue}

In die TypoView.vue Datei werden die Contentdaten der Typo3-Seite reingeladen. Sie bekommt diese als Property/Eigenschaft in der [...slug].vue Datei. Die Daten enthalten alle Elemente die auf der jeweiligen Seite im Typo3 gepflegt wurden. Diese sind in Form eines Arrays aufbereitet. Das erste Element im Array entspricht dabei dem im Typo3 ganz oben auf der Seite gepflegten Element. Das letzte Element entspricht dem ganz unten gepflegten Element. Dies bedeutet, dass wenn die Elemente aus dem Array in ihrer Reihenfolge dargestellt werden, dass dann alle Elemente an ihrem Platz sind. Damit die Daten aber genutzt werden können, müssen sie noch ein wenig aufbereitet werden. Die TypoView muss irgendwie wissen, welche Datei zur Darstellung der jeweiligen Typo3 Komponente zuständig ist. Dafür wurde ein Daten-Mapping der Komponenten erstellt. Dieses Daten-Mapping wurde in einem neuen Composable gespeichert, welches in der Datei componentMapping.ts vorhanden ist. In dem Daten-Mapping steht beispielsweise, dass ein Text-Element mit dem default Layout 'cms-components-rich-text' entspricht. In einer mapTypo3Content Funktion, wird dann durch den gesamten Array durch iteriert. In dieser Funktion wird aus dem eindimensionalen Array ein zweidimensionaler. Dieser zweidimensionale Array entspricht dann den Sections mit ihren jeweiligen Elementen. Dafür wird bei jedem Element abgefragt ob es eine Section ist, wenn dies der Fall ist, wird ein neuer Eintrag im zweidimensionalen Array erstellt. Wenn das nicht der Fall ist, wird das Element bei letzten Element-Array im Sections-Array angehängt. Zusätzlich wird das Daten-Mapping angewandt und dem Element mitgegeben. Die TypoView Komponente iteriert dann im Template durch den zweidimensionalen Array. Für jede Section wird die Section Komponente aufgerufen und für jedes Element in der Section die entsprechende Komponente. Dafür wird bei einem <component> Tag die Nuxt 3 eigene resolveComponent() Funktion genutzt. Ihr kann als Parameter ein String gegeben werden. Der String entspricht hier dem Wert, welcher aus dem Daten-Mapping kommt. Nuxt 3 überprüft dann, ob es eine Komponente mit dem Namen gibt und rendert diese dann.

\subsubsection{Section Komponente}
\label{sec:Section Komponente}

Die Section Komponente ist in dieser Iteration des Skeletons eine sehr einfache Komponente. Sie teilt lediglich die Website in Bereiche ein. Diese Bereiche können dann in zukünftigen Projekten genutzt werden. Ein klassische Anwendung wäre beispielsweise die Hintergrundfarbe zu ändern. So dass unterschiedliche Bereiche der Website hervorgehoben werden können. Theoretisch könnte in der Section auch Abstände gepflegt werden, oder unterschiedliche Scrollverhalten. Das alles würde hier aber den Rahmen der Arbeit sprengen und ist für ein Skeleton auch nicht notwendig. In ihrer aktuellen Iteration hat die Section Komponente lediglich einen Slot, indem ihre Elemente reingerendert werden.

\subsubsection{Content Komponenten}
\label{sec:Content Komponenten}

Für alle restlichen Komponenten die laut Projektantrag dargestellt werden sollen, wurde jeweils eine Komponente erstellt. Im Beispiel des Text-Elements wurde im components Verzeichnis im Unterverzeichnis cms-components die Datei RichText.vue angelegt. Diese kann nun im Template von Nuxt 3 durch <cms-components-rich-text> aufgerufen werden. Dafür benötigt es keinen Import. Analog dazu wurden für alle anderen Komponenten ebenfalls Dateien nach diesem Namensschema angelegt. Das Daten-Mapping der Komponenten wurde dann mit den Namen aufgefüllt.

\subsection{finale Anwendung}
\label{sec:finale Anwendung}

Screenshots der finalen Anwendung inklusive der gepflegten Dummy-Inhalte der Typo3 Seite befinden sich im \Anhang{Screenshots}.
