% !TEX root = ../Projektdokumentation.tex
\section{Implementierungsphase} 
\label{sec:Implementierungsphase}

\subsection{Ausspielen der Typo3 Daten}
\label{sec:Ausspielen der Typo3 Daten}

Damit die Typo3 Daten im JSON-Format ausgespielt werden, muss die Headless-Extension(https://extensions.typo3.org/extension/headless) installiert werden. Danach muss die Extension zum Template hinzugefügt werden, damit Typo3 weiß, dass es das Template dementsprechend anpassen muss. Typo3 spielt nun die Daten im JSON-Format aus. Das JSON muss noch angepasst werden, damit eine Navigationsstruktur damit abgebildet werden kann. Dafür werden folgende Anpassungen am Template vorgenommen. Abbildung


\subsection{Implementierung der Benutzeroberfläche}
\label{sec:ImplementierungBenutzeroberflaeche}

\begin{itemize}
	\item Beschreibung der Implementierung der Benutzeroberfläche, falls dies separat zur Implementierung der Geschäftslogik erfolgt (\zB bei \acs{HTML}-Oberflächen und Stylesheets).
	\item \Ggfs Beschreibung des Corporate Designs und dessen Umsetzung in der Anwendung.
	\item Screenshots der Anwendung
\end{itemize}

\paragraph{Beispiel}
Screenshots der Anwendung in der Entwicklungsphase mit Dummy-Daten befinden sich im \Anhang{Screenshots}.


\subsection{Implementierung der Geschäftslogik}
\label{sec:ImplementierungGeschaeftslogik}

\begin{itemize}
	\item Beschreibung des Vorgehens bei der Umsetzung/Programmierung der entworfenen Anwendung.
	\item \Ggfs interessante Funktionen/Algorithmen im Detail vorstellen, verwendete Entwurfsmuster zeigen.
	\item Quelltextbeispiele zeigen.
	\item Hinweis: Wie in Kapitel~\ref{sec:Einleitung}: \nameref{sec:Einleitung} zitiert, wird nicht ein lauffähiges Programm bewertet, sondern die Projektdurchführung. Dennoch würde ich immer Quelltextausschnitte zeigen, da sonst Zweifel an der tatsächlichen Leistung des Prüflings aufkommen können.
\end{itemize}

\paragraph{Beispiel}
Die Klasse \texttt{Com\-par\-ed\-Na\-tu\-ral\-Mo\-dule\-In\-for\-ma\-tion} findet sich im \Anhang{app:CNMI}.  
